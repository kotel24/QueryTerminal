\documentclass[a4paper,14pt]{extarticle} % 14 шрифт по ГОСТ

% --- Поддержка русского языка и шрифтов ---
\usepackage{fontspec}
\setmainfont{Times New Roman} % Шрифт текста
\setmonofont{Courier New}     % Шрифт кода
\usepackage{polyglossia}
\setdefaultlanguage{russian}

% --- Поля страницы по ГОСТ ---
\usepackage[left=3cm,right=1.5cm,top=2cm,bottom=2cm]{geometry}

% --- Межстрочный интервал 1.5 ---
\usepackage{setspace}
\onehalfspacing

% --- Пакеты для вставки картинок, ссылок и кода ---
\usepackage{graphicx}
\usepackage{hyperref}
\usepackage[outputdir=build]{minted} % Красивая подсветка кода (требует Python)
\usemintedstyle{vs} % Стиль подсветки Visual Studio

% --- Настройка библиографии (ГОСТ) ---
\usepackage[backend=biber,style=gost-numeric,sorting=none]{biblatex}
\addbibresource{references.bib}

\begin{document}

% ================= ТИТУЛЬНЫЙ ЛИСТ =================
\begin{titlepage}
    \centering
    \footnotesize
    МИНИСТЕРСТВО НАУКИ И ВЫСШЕГО ОБРАЗОВАНИЯ РОССИЙСКОЙ ФЕДЕРАЦИИ\\
    ФЕДЕРАЛЬНОЕ ГОСУДАРСТВЕННОЕ БЮДЖЕТНОЕ ОБРАЗОВАТЕЛЬНОЕ УЧРЕЖДЕНИЕ\\
    ВЫСШЕГО ОБРАЗОВАНИЯ\\
    \textbf{«МИРЭА – Российский технологический университет»}\\
    \vspace{0.5cm}
    Институт информационных технологий\\
    Кафедра инструментального и прикладного программного обеспечения\\

    \vspace{3cm}
    \Large \textbf{ОТЧЕТ ПО ПРАКТИЧЕСКОЙ РАБОТЕ №7}\\
    \normalsize по дисциплине «Инструментальные средства разработки ПО»\\
    \vspace{1cm}
    \Large \textbf{Тема: Интеграция средств автоматизированного документирования}\\

    \vspace{4cm}
    \begin{flushright}
        \begin{minipage}{0.4\textwidth}
            \flushright
            Выполнил студент:\\
            Котельников Д.В.\\
            Группа ИКБО-31-24\\
            \\
            Проверил:\\
            Шутов К.И.
        \end{minipage}
    \end{flushright}

    \vfill
    Москва 2025
\end{titlepage}

% ================= СОДЕРЖАНИЕ =================
\tableofcontents
\newpage

% ================= ВВЕДЕНИЕ =================
\section{Введение}
Целью данной работы является изучение инструментов документирования кода (Doxygen) и издательской системы \LaTeX{}. В качестве объекта исследования используется модуль \texttt{qt.py}, реализующий работу с базой данных SQLite.

% ================= ПОДКЛЮЧЕНИЕ ВАШЕЙ ИНСТРУКЦИИ =================
\section{Руководство пользователя}

\subsection{Общие сведения}
Библиотека \textbf{QT Database Wrapper} предназначена для упрощения работы с SQLite...

\subsection{Установка}
Для установки выполните команду:
\begin{minted}{bash}
pip install pytest
\end{minted}

\subsection{Примеры использования}
Инициализация базы данных:
\begin{minted}{python}
from qt import QT
db = QT(":memory:")
\end{minted}

% ================= АНАЛИЗ КОДА =================
\section{Анализ программного кода}
Для выполнения работы был проанализирован основной файл курсовой работы (см. Листинг 1).

\begin{listing}[H]
\caption{Код из qt.py}
\inputminted[firstline=1, lastline=314, linenos]{python}{qt.py}
\end{listing}

% ================= ЗАКЛЮЧЕНИЕ =================
\section{Заключение}
В ходе практической работы была сформирована документация API и сверстан отчет в системе \LaTeX{}.

% ================= СПИСОК ЛИТЕРАТУРЫ =================
\printbibliography[heading=bibintoc, title={Список использованных источников}]

\end{document}